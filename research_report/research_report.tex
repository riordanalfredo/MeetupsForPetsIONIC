\documentclass[conference]{IEEEtran}
\IEEEoverridecommandlockouts
% The preceding line is only needed to identify funding in the first footnote. If that is unneeded, please comment it out.
\usepackage{cite}
\usepackage{amsmath,amssymb,amsfonts}
\usepackage{algorithmic}
\usepackage{graphicx}
\usepackage{textcomp}
\usepackage{xcolor}
\usepackage{hyperref}
\usepackage{float}
\usepackage{subfig}

\def\BibTeX{{\rm B\kern-.05em{\sc i\kern-.025em b}\kern-.08em
    T\kern-.1667em\lower.7ex\hbox{E}\kern-.125emX}}
\begin{document}

\title{Model Driven Approach for Migration Problem in Hybrid App Development\\}

%TODO: Insert names for each team member%
\author{\IEEEauthorblockN{1\textsuperscript{st} Given Name Surname}
\IEEEauthorblockA{\textit{dept. name of organization (of Aff.)} \\
\textit{name of organization (of Aff.)}\\
City, Country \\
email address or ORCID}
\and
\IEEEauthorblockN{2\textsuperscript{nd} Given Name Surname}
\IEEEauthorblockA{\textit{dept. name of organization (of Aff.)} \\
\textit{name of organization (of Aff.)}\\
City, Country \\
email address or ORCID}
\and
\IEEEauthorblockN{Riordan Dervin Alfredo}
\IEEEauthorblockA{\textit{Faculty of IT, Monash University} \\
Melbourne, Australia \\
riordan.alfredo@gmail.com}
\and
\IEEEauthorblockN{Mubtasim Mahmud}
\IEEEauthorblockA{\textit{Faculty of IT, Monash University} \\
Melbourne, Australia \\
mubtasimmahmud20@gmail.com}
}

\maketitle

\begin{abstract}
This document is a model and instructions for \LaTeX.
This and the IEEEtran.cls file define the components of your paper [title, text, heads, etc.]. *CRITICAL: Do Not Use Symbols, Special Characters, Footnotes,
or Math in Paper Title or Abstract.
\end{abstract}

\begin{IEEEkeywords}
component, formatting, style, styling, insert
\end{IEEEkeywords}

\section{Introduction}
This document is a model and instructions for \LaTeX.
Please observe the conference page limits.

\section{Approach}

\subsection{Case Study}

The IEEEtran class file is used to format your paper and style the text. All margins,
column widths, line spaces, and text fonts are prescribed; please do not
alter them. You may note peculiarities. For example, the head margin
measures proportionately more than is customary. This measurement
and others are deliberate, using specifications that anticipate your paper
as one part of the entire proceedings, and not as an independent document.
Please do not revise any of the current designations.


\subsection{The Modelling Tool}

\section{Evaluation}

The results of the migration tool we developed were quite favorable. The case study
involved migrating three categories of changes:
\begin{enumerate}
    \item HTML markup
    \item Typescript code
    \item Architectural changes (file and folder structure)
\end{enumerate}

\subsection{HTML Markup}
The migration tool tackles the problem by separating the full
text into components (.i.e buttons, labels, etc.). This is the most
general category, as the migration process for different components
has a lot of common procedures.

\begin{figure}
    \centering
    \subfloat[Ionic 3]{
        \includegraphics[width=0.8\linewidth]{button.PNG}
        \label{fig:ion3button}
    }
    \qquad
    \subfloat[Ionic 4]{
        \includegraphics[width=0.8\linewidth]{buttonv4.PNG}
        \label{fig:ion4button}
    }
    \caption{Ionic 3-4 Button Migration (Markup)}
    \label{fig:ionicButtonMigration}
\end{figure}

\subsection{Typescript code}
Migration of Typescript code involves looking through code files
to find specific tokens and changing them to fit the Ionic 4 specification.
Unlike HTML markup, every different migration process in this category is almost
completely different, and can only be generalized in broad terms, such as find-and-replace
functionality.

\begin{figure}
    \centering
    \subfloat[Ionic 3]{
        \includegraphics[width=0.8\linewidth]{lifecycle.PNG}
        \label{fig:lifecycle3}
    }
    \qquad
    \subfloat[Ionic 4]{
        \includegraphics[width=0.8\linewidth]{lifecyclev4.PNG}
        \label{fig:lifecycle4}
    }
    \caption{Ionic 3-4 Lifecycle Methods Migration (Typescript)}
    \label{fig:ionicLifecycleMigration}
\end{figure}

\subsection{Architectural Changes}
The structure of an Ionic 4 project is drastically different from an Ionic 3 project,
making its migration a very specific process. The tool focuses on a small subset, which
involves reorganizing files associated with one specific feature of an app to the equivalent Ionic 4
folder structure, as well as renaming them to fit the new convention.

\begin{figure}
    \centering
    \subfloat[Ionic Feature Migration Output]{
        \includegraphics[width=0.8\linewidth]{files.PNG}
        \label{fig:ionFiles}
    }
    \caption{Ionic 3-4 File Rename-and-Relocate Migration (Architectural)}
    \label{fig:ionicFileMigration}
\end{figure}


\section{Discussion/Experiences}
The overall consensus is that model-driven development is a feasible approach
for software migration, especially regarding this case study in particular.
\newline \newline
For HTML markup in particular, automating component-wise migration is remarkably simple, and it
is well within the scope of possibility to create a general function that will migrate any Ionic
3 component to its Ionic 4 equivalent, provided a library of changes specific to each component
is collated for use by this function.
\newline \newline
Typescript code can be a little more complicated to deal with. During development of the tool,
we encountered some difficulty with porting over two components in particular - \textbf{Toasts} and
\textbf{Alerts}. The difficulty stems from the asynchronous nature of the two in Ionic 4 as opposed to their
counterparts in Ionic 3 - handler function callbacks and deeply nested toasts and alerts have to be handled differently,
which makes migrating these a more involved process than just pure automation.
\newline \newline
Architectural changes vary in difficulty. A project-wide relocation and renaming of files is well within the
scope of possibility, but changes to the routing mechanisms adopted in the different versions are drastic and
require significant changes to the codebase. Ionic 4 uses the routing functionality provided by the frontend framework
specified when creating a project using its command-line interface, meaning that a separate migration process would need
to be adopted for, say, React vs. Angular, for example. Ionic 3 used its own proprietary router on the other hand, which means
at least that part of the code is easier to deal with. While Ionic 4 adopted this change in order to be future-proof, this does
make a generalized migration tool for this specific feature a lot harder to implement.


\section{Related Work}
Number equations consecutively. To make your
equations more compact, you may use the solidus (~/~), the exp function, or
appropriate exponents. Italicize Roman symbols for quantities and variables,
but not Greek symbols. Use a long dash rather than a hyphen for a minus
sign. Punctuate equations with commas or periods when they are part of a
sentence, as in:
\begin{equation}
a+b=\gamma\label{eq}
\end{equation}

Be sure that the
symbols in your equation have been defined before or immediately following
the equation. Use ``\eqref{eq}'', not ``Eq.~\eqref{eq}'' or ``equation \eqref{eq}'', except at
the beginning of a sentence: ``Equation \eqref{eq} is . . .''


\section{Conclusion}
Number equations consecutively. To make your
equations more compact, you may use the solidus (~/~), the exp function, or
appropriate exponents. Italicize Roman symbols for quantities and variables,
but not Greek symbols. Use a long dash rather than a hyphen for a minus
sign. Punctuate equations with commas or periods when they are part of a
sentence, as in:
\begin{equation}
a+b=\gamma\label{eq}
\end{equation}

Be sure that the
symbols in your equation have been defined before or immediately following
the equation. Use ``\eqref{eq}'', not ``Eq.~\eqref{eq}'' or ``equation \eqref{eq}'', except at
the beginning of a sentence: ``Equation \eqref{eq} is . . .''

\subsection{\LaTeX-Specific Advice}

Please use ``soft'' (e.g., \verb|\eqref{Eq}|) cross references instead
of ``hard'' references (e.g., \verb|(1)|). That will make it possible
to combine sections, add equations, or change the order of figures or
citations without having to go through the file line by line.

Please don't use the \verb|{eqnarray}| equation environment. Use
\verb|{align}| or \verb|{IEEEeqnarray}| instead. The \verb|{eqnarray}|
environment leaves unsightly spaces around relation symbols.

Please note that the \verb|{subequations}| environment in {\LaTeX}
will increment the main equation counter even when there are no
equation numbers displayed. If you forget that, you might write an
article in which the equation numbers skip from (17) to (20), causing
the copy editors to wonder if you've discovered a new method of
counting.

{\BibTeX} does not work by magic. It doesn't get the bibliographic
data from thin air but from .bib files. If you use {\BibTeX} to produce a
bibliography you must send the .bib files.

{\LaTeX} can't read your mind. If you assign the same label to a
subsubsection and a table, you might find that Table I has been cross
referenced as Table IV-B3.

{\LaTeX} does not have precognitive abilities. If you put a
\verb|\label| command before the command that updates the counter it's
supposed to be using, the label will pick up the last counter to be
cross referenced instead. In particular, a \verb|\label| command
should not go before the caption of a figure or a table.

Do not use \verb|\nonumber| inside the \verb|{array}| environment. It
will not stop equation numbers inside \verb|{array}| (there won't be
any anyway) and it might stop a wanted equation number in the
surrounding equation.

\subsection{Some Common Mistakes}\label{SCM}
\begin{itemize}
\item The word ``data'' is plural, not singular.
\item The subscript for the permeability of vacuum $\mu_{0}$, and other common scientific constants, is zero with subscript formatting, not a lowercase letter ``o''.
\item In American English, commas, semicolons, periods, question and exclamation marks are located within quotation marks only when a complete thought or name is cited, such as a title or full quotation. When quotation marks are used, instead of a bold or italic typeface, to highlight a word or phrase, punctuation should appear outside of the quotation marks. A parenthetical phrase or statement at the end of a sentence is punctuated outside of the closing parenthesis (like this). (A parenthetical sentence is punctuated within the parentheses.)
\item A graph within a graph is an ``inset'', not an ``insert''. The word alternatively is preferred to the word ``alternately'' (unless you really mean something that alternates).
\item Do not use the word ``essentially'' to mean ``approximately'' or ``effectively''.
\item In your paper title, if the words ``that uses'' can accurately replace the word ``using'', capitalize the ``u''; if not, keep using lower-cased.
\item Be aware of the different meanings of the homophones ``affect'' and ``effect'', ``complement'' and ``compliment'', ``discreet'' and ``discrete'', ``principal'' and ``principle''.
\item Do not confuse ``imply'' and ``infer''.
\item The prefix ``non'' is not a word; it should be joined to the word it modifies, usually without a hyphen.
\item There is no period after the ``et'' in the Latin abbreviation ``et al.''.
\item The abbreviation ``i.e.'' means ``that is'', and the abbreviation ``e.g.'' means ``for example''.
\end{itemize}
An excellent style manual for science writers is \cite{b7}.

\subsection{Authors and Affiliations}
\textbf{The class file is designed for, but not limited to, six authors.} A
minimum of one author is required for all conference articles. Author names
should be listed starting from left to right and then moving down to the
next line. This is the author sequence that will be used in future citations
and by indexing services. Names should not be listed in columns nor group by
affiliation. Please keep your affiliations as succinct as possible (for
example, do not differentiate among departments of the same organization).

\subsection{Identify the Headings}
Headings, or heads, are organizational devices that guide the reader through
your paper. There are two types: component heads and text heads.

Component heads identify the different components of your paper and are not
topically subordinate to each other. Examples include Acknowledgments and
References and, for these, the correct style to use is ``Heading 5''. Use
``figure caption'' for your Figure captions, and ``table head'' for your
table title. Run-in heads, such as ``Abstract'', will require you to apply a
style (in this case, italic) in addition to the style provided by the drop
down menu to differentiate the head from the text.

Text heads organize the topics on a relational, hierarchical basis. For
example, the paper title is the primary text head because all subsequent
material relates and elaborates on this one topic. If there are two or more
sub-topics, the next level head (uppercase Roman numerals) should be used
and, conversely, if there are not at least two sub-topics, then no subheads
should be introduced.

\subsection{Figures and Tables}
\paragraph{Positioning Figures and Tables} Place figures and tables at the top and
bottom of columns. Avoid placing them in the middle of columns. Large
figures and tables may span across both columns. Figure captions should be
below the figures; table heads should appear above the tables. Insert
figures and tables after they are cited in the text. Use the abbreviation
``Fig.~\ref{fig}'', even at the beginning of a sentence.
% Below is the method to create a table %
\begin{table}[htbp]
\caption{Table Type Styles}
\begin{center}
\begin{tabular}{|c|c|c|c|}
\hline
\textbf{Table}&\multicolumn{3}{|c|}{\textbf{Table Column Head}} \\
\cline{2-4}
\textbf{Head} & \textbf{\textit{Table column 1}}& \textbf{\textit{Subhead}}& \textbf{\textit{Subhead}} \\
\hline
copy& More table copy$^{\mathrm{a}}$& &  \\
\hline
\multicolumn{4}{l}{$^{\mathrm{a}}$Sample of a Table footnote.}
\end{tabular}
\label{tab1}
\end{center}
\end{table}

\begin{figure}[htbp]
\centerline{\includegraphics{fig1.png}}
\caption{Example of a figure caption.}
\label{fig}
\end{figure}

Figure Labels: Use 8 point Times New Roman for Figure labels. Use words
rather than symbols or abbreviations when writing Figure axis labels to
avoid confusing the reader. As an example, write the quantity
``Magnetization'', or ``Magnetization, M'', not just ``M''. If including
units in the label, present them within parentheses. Do not label axes only
with units. In the example, write ``Magnetization (A/m)'' or ``Magnetization
\{A[m(1)]\}'', not just ``A/m''. Do not label axes with a ratio of
quantities and units. For example, write ``Temperature (K)'', not
``Temperature/K''.

\section*{Acknowledgment}

The preferred spelling of the word ``acknowledgment'' in America is without
an ``e'' after the ``g''. Avoid the stilted expression ``one of us (R. B.
G.) thanks $\ldots$''. Instead, try ``R. B. G. thanks$\ldots$''. Put sponsor
acknowledgments in the unnumbered footnote on the first page.
\section*{References}

Please number citations consecutively within brackets \cite{b1}. The
sentence punctuation follows the bracket \cite{b2}. Refer simply to the reference
number, as in \cite{b3}---do not use ``Ref. \cite{b3}'' or ``reference \cite{b3}'' except at
the beginning of a sentence: ``Reference \cite{b3} was the first $\ldots$''

Number footnotes separately in superscripts. Place the actual footnote at
the bottom of the column in which it was cited. Do not put footnotes in the
abstract or reference list. Use letters for table footnotes.

Unless there are six authors or more give all authors' names; do not use
``et al.''. Papers that have not been published, even if they have been
submitted for publication, should be cited as ``unpublished'' \cite{b4}. Papers
that have been accepted for publication should be cited as ``in press'' \cite{b5}.
Capitalize only the first word in a paper title, except for proper nouns and
element symbols.

For papers published in translation journals, please give the English
citation first, followed by the original foreign-language citation \cite{b6}.

% @Author: Riordan Dervin Alfredo, 2/11/2019 %
\section{Related Work}
In regards to the software migration that utilises model driven engineering methods,
there are already several studies that were conducted extensively before.
Those are; FASMM:
Fast and Accessible Software Migration Method \cite{b2}, Reverse Engineering
Strategies for Software Migration \cite{b6}, Model-Driven Engineering
for Software Migration in a Large Industrial Context \cite{b3},
Towards a Model-Driven Approach for Planning a
Standard-Based Migration of Enterprise Applications to SOA  \cite{b5}, and
Migrating C/C++ Software to Mobile Platforms in the ADM Context \cite{b4}

Our study is closely related to the FASMM approaches \cite{b2} as this research
provides extensive guides for software developers that have limited
resources (time, budget, workforce $\ldots$) to conduct software migration.
Also, \cite{b6} study provides specific methods to conduct reverse-engineering
that is involved our process.

In \cite{b3} and \cite{b5} studies are proving the feasibility of model-driven engineering
in diffent domain and contexts from our studies. Those are the migration to another completely new framework,
migration to a new architecture (legacy achitecture to service-oriented architecture), and migration
of different languages to specific mobile applications framework.
They also provided insights of how model-driven engineering processes to be economically
profitable and provides cost-effectiveness.

The \cite{b4} study related to our work in terms of processing different kind of
programming languages and paradigms to generate mobile application codes. It is also
the main example of feasibility model-driven in the software migration of mobile applications,
which is in our case is software migration of the hybrid mobile application.

Each related work are explained in details below:
\subsection{ FASMM: Fast and Accessible Software Migration Method }

\subsection{ Reverse Engineering Strategies for Software Migration }

\subsection{ Model-Driven Engineering for Software Migration in a Large Industrial Context }
This paper describes the process of migrating a large-scale software application from Mainframe to J2EE
using Model-Driven Engineering process. The migration process in this study starts by describing the general
processes of current legacy system.

First, a parser is created to make an abstract systax tree from the legacy code. Then, it is processed
by a transformation to build a model that conforms to the meta-model of the legacy language.
During the second stage, all symbols are resolved and bound to appropriate model elements.
Afterwards, from code model, reverse-engineering process is conducted to produce a platform independent
model. Finally, this model will be mapped into a platform specific model that can be used to generate code
for the final product of the migrated application. Below are the general picture of process that
undertakes the similar process as this research study:

\begin{figure}[htbp]
\centerline{\includegraphics{fig2.png}}
\caption{Model-driven migration principle \cite{b3}}
\label{fig}
\end{figure}

\subsection{ Towards a Model-Driven Approach for Planning a Standard-Based Migration of Enterprise Applications to SOA }

\subsection{ Migrating C/C++ Software to Mobile Platforms in the ADM Context }
go to here \url{ https://pdfs.semanticscholar.org/8968/7958518252817c4bbe02c77fcef5a48a8d3c.pdf }

This study took an example of  C/C++ language to produce Android, IoS, and Windows mobile application.
In our study, the extension type of generated codes are still maintained,
but the main code-generation process is technically alike .

Furthermore, this study gives insight in regards to feasibility of model-driven development approach that is
advanced to ADM (Architecture-Driven Modernization) within similar domain, mobile application migration.
ADM approach that are described in this study could potentially be part of the future works in our study.
The validation tool that is used in this research is to our approach, which is Eclipse Modeling Framework.
mobile applications from different kind of programming languages, which is technically
how hybrid application system / architecture works.


\section{Conclusion}

In conclusion, $\ldots$
There are still many challenges to be taken down that produce the proposed method for
hybrid android  migration

Talk about Future works here:


\begin{thebibliography}{00}
\bibitem{b1} G. Eason, B. Noble, and I. N. Sneddon, ``On certain integrals of Lipschitz-Hankel type involving products of Bessel functions,'' Phil. Trans. Roy. Soc. London, vol. A247, pp. 529--551, April 1955.
\bibitem{b2} J. Clerk Maxwell, A Treatise on Electricity and Magnetism, 3rd ed., vol. 2. Oxford: Clarendon, 1892, pp.68--73.
\bibitem{b3} I. S. Jacobs and C. P. Bean, ``Fine particles, thin films and exchange anisotropy,'' in Magnetism, vol. III, G. T. Rado and H. Suhl, Eds. New York: Academic, 1963, pp. 271--350.
\bibitem{b4} K. Elissa, ``Title of paper if known,'' unpublished.
\bibitem{b5} R. Nicole, ``Title of paper with only first word capitalized,'' J. Name Stand. Abbrev., in press.
\bibitem{b6} Y. Yorozu, M. Hirano, K. Oka, and Y. Tagawa, ``Electron spectroscopy studies on magneto-optical media and plastic substrate interface,'' IEEE Transl. J. Magn. Japan, vol. 2, pp. 740--741, August 1987 [Digests 9th Annual Conf. Magnetics Japan, p. 301, 1982].
\bibitem{b7} M. Young, The Technical Writer's Handbook. Mill Valley, CA: University Science, 1989.
\end{thebibliography}
\vspace{12pt}
\color{red}
IEEE conference templates contain guidance text for composing and formatting conference papers. Please ensure that all template text is removed from your conference paper prior to submission to the conference. Failure to remove the template text from your paper may result in your paper not being published.

\end{document}
